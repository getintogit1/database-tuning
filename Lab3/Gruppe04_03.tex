\documentclass[11pt]{scrartcl}

\usepackage[top=1.5cm]{geometry}
\usepackage{float}
\usepackage{listings}
\usepackage{xcolor}

\setlength{\parindent}{0em}
\setlength{\parskip}{0.5em}

\newcommand{\youranswerhere}{[Your answer goes here \ldots]}
\renewcommand{\thesubsection}{\arabic{subsection}}

\lstdefinestyle{dbtsql}{
  language=SQL,
  basicstyle=\small\ttfamily,
  keywordstyle=\color{magenta!75!black},
  stringstyle=\color{green!50!black},
  showspaces=false,
  showstringspaces=false,
  commentstyle=\color{gray}}

\title{
  \textbf{\large Assignment 3} \\
  Index Tuning \\
  {\large Database Tuning}}

\author{
  Group Name (e.g. A1, B5, B3) \\
  \large Balint Peter, 12213073 \\
  \large Günter Lukas, 12125639 \\
  \large Ottino David, 51841010
}

\begin{document}

\maketitle

\textbf{Database system and version:} \emph{PostgreSQL 17.4-1}

\subsection{Index Data Structures}

Which index data structures (e.g., B$^+$ tree index) are supported?

\subsubsection{B-Tree}
Postgres uses B-Trees as it's standard data structure for storing indexes. Whenever the \texttt{CREATE INDEX} command is used without additional parameters, a B-Tree is created. The main difference between a B-tree and a B$^+$ tree is that with a B-tree the data is stored in either internal or leaf nodes while a B$^+$ tree stores the data in leaf nodes only. Intuitively, this may seem more efficient , but  this also leads to more storage being used for one internal node which decreases the amount of pointers you can pack into into it. This leads to a smaller fanout compared to a B$^+$ tree. Additionally,  the B$^+$ tree also uses a linked list for data stored in leaf nodes which is not possible for a B tree. Generally speaking B trees are more efficient when it comes to single access of random data while B$^+$ trees are prefered when range-queries have to be executed.

\subsubsection{Hash}
Hash indexes in Postgres store a 32-bit hash code of the indexed column which is then used to find and access data. Due to its nature hash indexes are considered when the indexed column is involved in a comparison using the "-"-Operator.

\subsubsection{GiST}
GiST stands for Generalized Search Tree. It doesn't describe a specific index type but is more of a framework which allows Postgres-Users to implement their own arbitrary indexing schemes, for example B$^+$ trees. However, it comes with some built-in standard functionality.

\subsubsection{SP-GiST}
The "SP" in this index type stands for space-partitioned GiST, which means that the search space is repeatedly divided into different partitions of potentially different sizes. Searches that are well matched to the partitioning rule can be very efficient. Similar to GiST, SP-GiST allows users to develop their own custom data types and access methods.

\subsubsection{GIN}
GIN stands for "Generalized Inverted Index" and is used for complex data values that consist of different components, such as arrays or texts, and the expected queries search for different elements within this data. GIN stores each of these components, so called keys, within a separate index. Each key references to the tupels it is contained in. It is also possible to develop custom data types and access methods with this index type.

\subsubsection{BRIN}
BRIN, which is shorthand for "Block Range Indexes", is mainly used for very large tables in which attributes can naturally be ordered in some way and this order is resembled by their location in physical memory. BRIN stores a summary of all the data present in one or multiple pages that are besides each other in memory. This means that the index itself is very small and therefore fast to traverse. However, BRIN is a lossy type of index, which means that after traversing the index, the returned tupels will still have to be checked for the actual selection criteria.

\subsection{Clustering Indexes}

Discuss how the system supports clustering indexes, in particular:

\paragraph{a)}

How do you create a clustering index on \texttt{ssnum}? Show the query.\footnote{Give the queries for creating a hash index \emph{and} a B$^+$ tree index if both of them are supported.}

PostgreSQL does not directly provide clustered indexes. However, the command \texttt{CLUSTER} clusters a given table based on an already existing index, but this doesn't work on a hash index.

\begin{lstlisting}[style=dbtsql]
CREATE INDEX idx_employee_ssnum ON Employee(ssnum);
CLUSTER Employee USING idx_employee_ssnum;
\end{lstlisting}

\paragraph{b)}

Are clustering indexes on non-key attributes supported, e.g., on \texttt{name}? Show the query.

Yes, it is possible to create clustering indexes on all attributes using the aforementioned method.

\begin{lstlisting}[style=dbtsql]
CREATE INDEX IDX_EMPLOYEE_NAME ON EMPLOYEE (NAME);
CLUSTER EMPLOYEE USING IDX_EMPLOYEE_NAME;
\end{lstlisting}

\paragraph{c)}

Is the clustering index dense or sparse?

The PostgreSQL-documentation does not outright state if the created indexes are dense or sparse. However, it is possible to view the number of entries of a given index which allows us to conclude if the index is dense or sparse. 

\begin{lstlisting}[style=dbtsql]
SELECT COUNT(DISTINCT MANAGER)FROM EMPLOYEE;
CREATE INDEX IDX_EMPLOYEE_MANAGER ON EMPLOYEE (MANAGER);
SELECT reltuples FROM PG_CLASS
WHERE RELNAME = 'idx_employee_manager';
\end{lstlisting}

We know from the first query that there are 49.825 managers. But the total number of tuples in the index is 100.000, which is the total amount of rows. Therefore we can conclude that the index is dense.

\paragraph{d)}

How does the system deal with overflows in clustering indexes? How is the fill factor controlled?

Since the clustering index in PostgreSQL is not really a clustered index in the traditional sense, overflows are not really an issue since new data is just inserted whereever there is an open space and a repeated clustering based on the index leads to the whole data structure being reorganized.
The fill factor can be set by the user for each table or index individually with the following statements:

\begin{lstlisting}[style=dbtsql]
ALTER TABLE EMPLOYEE SET (fillfactor = 80);
CREATE INDEX IDX_EMPLOYEE_MANAGER ON EMPLOYEE (MANAGER) WITH (fillfactor = 80);
\end{lstlisting}

The specified fill factor is stored in \texttt{pg\_class} in the \texttt{reloptions}-attribute.

\paragraph{e)}

If new data is inserted into the table, the additions won't be clustered automatically and a new \texttt{CLUSTER}-operation has to be executed. Clustering took about 1,5 seconds in our case.

\begin{lstlisting}[style=dbtsql]
CLUSTER Employee;
\end{lstlisting}

The \texttt{CLUSTER}-operation without any arguemnts will cluster according to the last index used when clustering.

\youranswerhere{}

\subsection{Non-Clustering Indexes}

Discuss how the system supports non-clustering indexes, in particular:


\paragraph{a)}

How do you create a combined, non-clustering index on \texttt{(dept,salary)}? Show the query.$^1$

\youranswerhere{}

In PostgreSQL, indexes are by default non-clustered unless explicitly specified otherwise. To create a combined (i.e., multi-column) index on \texttt{(dept, salary)}, the following SQL statement can be used:

\begin{lstlisting}[style=dbtsql]
CREATE NONCLUSTERED INDEX idx_dept_salary
ON employee (dept, salary);
\end{lstlisting}


\paragraph{b)}

\textbf{Index-only Scans} sind ein vergleichsweise neues Feature in PostgreSQL und wurden erst mit Version 9.2 eingeführt. Der Grund für die späte Einführung liegt darin, dass PostgreSQL-Indizes von Haus aus keine Informationen über die Sichtbarkeit von Tupeln enthalten. Daher musste bisher immer zusätzlich auf die eigentliche Tabelle zugegriffen werden, um festzustellen, welche Tupel für eine bestimmte Transaktion sichtbar sind.

Mit der Einführung von Index-only Scans nutzt PostgreSQL nun die sogenannte \textit{Visibility Map}. Diese speichert für jede Datenseite ein einzelnes Bit, das angibt, ob alle Tupel auf dieser Seite für alle aktiven Transaktionen sichtbar sind. Ist dieses Bit gesetzt, kann auf einen direkten Zugriff auf die Tabelle verzichtet werden – der Index allein reicht aus, um das Ergebnis zu liefern. Voraussetzung dafür ist natürlich, dass alle für die Abfrage benötigten Daten bereits im Index enthalten sind.

In der Praxis bedeutet das jedoch nicht, dass niemals auf die Tabelle zugegriffen wird. Wenn das Visibility-Bit für eine Seite nicht gesetzt ist, muss der Datensatz weiterhin aus der Tabelle gelesen werden. Daher ist der Begriff „Index-only Scan“ etwas irreführend – treffender wäre wohl „Index-mostly Scan“.

Damit PostgreSQL überhaupt einen Index-only Scan in Betracht zieht, muss ein Großteil der Visibility-Bits gesetzt sein. Aus diesem Grund ist dieses Feature vor allem für statische oder selten veränderte Tabellen besonders geeignet.

\vspace{0.5em}
\noindent\textbf{Covered Query (Covering Index):}

Im Folgenden zeigen wir ein Beispiel für die Verwendung eines sogenannten \textit{Covering Indexes}. Ziel ist es, das \textbf{Durchschnittsgehalt} aller Angestellten zu ermitteln, die in der Abteilung \texttt{'TechdeptA'} arbeiten.

Zunächst wird ein Index auf den Spalten \texttt{(dept, salary)} der Tabelle \texttt{employee} erstellt, danach erfolgt die eigentliche Abfrage:

\begin{lstlisting}[style=dbtsql]
//Create B-Tree index on employee(dept, salary).
CREATE INDEX s_idx ON employee(dept, salary);

//Execute the actual query.
EXPLAIN (ANALYZE, BUFFERS)
SELECT AVG(salary)
FROM employee
WHERE dept = 'TechdeptA';
\end{lstlisting}

PostgreSQL liefert dabei den folgenden Ausführungsplan:

\begin{verbatim}
Aggregate  (cost=42.22..42.23 rows=1 width=4)
  (actual time=0.305..0.305 rows=1 loops=1)
  Buffers: shared hit=7
  -> Index Only Scan using s_idx on employee e 
     (cost=0.42..39.49 rows=1090 width=4)
     (actual time=0.108..0.236 rows=1002 loops=1)
     Index Cond: (dept = 'TechdeptA'::text)
     Heap Fetches: 0
     Buffers: shared hit=7
Total runtime: 0.348 ms
\end{verbatim}

Der Index wird hier erfolgreich verwendet, um die Anfrage ausschließlich über die im Index enthaltenen Informationen zu beantworten. Die Tabelle selbst wird nicht gelesen – die Anzahl der Heap-Fetches ist also \texttt{0}.

\vspace{0.5em}
\noindent\textbf{Covered Query, but not prefix:}

Nun soll die \textbf{Anzahl aller Angestellten} ermittelt werden, die ein Gehalt unter \texttt{2000} erhalten. Obwohl das Attribut \texttt{salary} im Index vorhanden ist, ist der Index primär nach \texttt{dept} geordnet.

\begin{lstlisting}[style=dbtsql]
SELECT COUNT(*)
FROM employee
WHERE salary < 2000;
\end{lstlisting}

Ausführungsplan:

\begin{verbatim}
Aggregate  (cost=2207.90..2207.91 rows=1 width=0)
  -> Seq Scan on employee e  
     (cost=0.00..2084.01 rows=49556 width=0)
     Filter: (salary < 2000)
\end{verbatim}

Hier entscheidet sich der PostgreSQL-Query-Planner gegen die Nutzung des bestehenden Indexes auf \texttt{(dept, salary)}. Stattdessen wird ein \texttt{Sequential Scan} durchgeführt. Der Grund: Die Bedingung bezieht sich nicht auf den führenden Teil (Prefix) des Indexes, und der Planner schätzt, dass der Index in diesem Fall keine Vorteile bringt.


\paragraph{c)}

\textbf{Weitere relevante Eigenschaften von Non-Clustering Indexes in PostgreSQL für das Datenbank-Tuning:}

PostgreSQL bietet neben klassischen, nicht-clusternden (non-clustered) Indizes zwei besonders interessante Funktionen, die die Effizienz von Index-Nutzung weiter steigern können: \textit{Partial Indexes} (partielle Indizes) und \textit{Indexes on Expressions} (Indizes auf Ausdrücke).

\vspace{0.5em}
\noindent\textbf{Partial Indexes (Partielle Indizes):}

Ein partieller Index wird nur über eine Teilmenge der Tabelle erstellt. Die zu indexierenden Tupel werden über eine \texttt{WHERE}-Klausel definiert. Diese Technik eignet sich insbesondere, um sehr häufig vorkommende Werte auszuschließen, die bei Abfragen ohnehin nicht vom Index profitieren würden.

Die zugrunde liegende Idee ist, dass der PostgreSQL-Query-Planner bei sehr häufigen Werten tendenziell auf einen \texttt{Sequential Scan} zurückgreift. Ein partieller Index auf nur selektive Werte bietet daher mehrere Vorteile:

\begin{itemize}
  \item Der Index bleibt klein und speichereffizient.
  \item Abfragen, die den Index nutzen, können schneller ausgeführt werden.
  \item Der Wartungsaufwand bei Aktualisierungen oder Einfügungen reduziert sich.
\end{itemize}

Allerdings sollte der mögliche Performance-Gewinn nicht überschätzt werden – in vielen praktischen Fällen bleibt der Effekt moderat.

\vspace{0.5em}
\noindent\textbf{Indexes on Expressions (Indizes auf Ausdrücke):}

Mit dieser Funktion kann PostgreSQL auch Ausdrücke oder Funktionswerte als Grundlage für einen Index verwenden. Das ist besonders nützlich, wenn in Abfragen Funktionen wie \texttt{LOWER()}, \texttt{ABS()}, \texttt{DATE\_TRUNC()} oder mathematische Operationen verwendet werden.

\begin{lstlisting}[style=dbtsql]
// Beispielabfrage:
SELECT * FROM test1 WHERE lower(col1) = 'value';

// Entsprechender Index:
CREATE INDEX test1_lower_col1_idx ON test1 (lower(col1));
\end{lstlisting}

In diesem Beispiel würde normalerweise bei jeder Zeile die Funktion \texttt{lower()} zur Laufzeit aufgerufen. Durch den Index auf \texttt{lower(col1)} wird dieser Schritt bereits vorab gespeichert. Dadurch kann PostgreSQL die Anfrage effizienter beantworten – der Funktionsaufruf entfällt, und der Vergleich erfolgt direkt im Index.

Solche Ausdrucksindizes sind besonders effektiv bei häufig wiederkehrenden Transformationen oder bei Funktionen, die nicht index-transparent sind, wie z.\,B. Groß-/Kleinschreibung oder String-Manipulation.

\vspace{0.5em}
\noindent\textbf{Fazit:}

Für das Datenbank-Tuning sind partielle Indizes und Ausdrucksindizes leistungsstarke Werkzeuge. Sie ermöglichen eine gezielte Optimierung des Indexzugriffs, reduzieren I/O-Kosten und können den Abfrageplan spürbar verbessern – besonders in großen, stark frequentierten Systemen.

\subsection{Key Compression and Page Size}

If your system supports B$^+$ trees, what kind of key compression (if any) is supported? How large is the default disk page? Can it be changed?

\youranswerhere{}

\subsection*{Time Spent on this Assignment}

Time in hours per person: \textbf{XXX}

\subsection*{References}

\begin{table}[H]
  \centering
  \begin{tabular}{c}
    \hline
    \textbf{Important:} Reference your information sources! \tabularnewline
    https://www.postgresql.org/docs/current/indexes-types.html \tabularnewline
    https://www.shiksha.com/online-courses/articles/difference-between-b-tree-and-b-plus-tree-blogId-155905 \tabularnewline
    https://www.postgresql.org/docs/current/gin.html \tabularnewline
    https://www.postgresql.org/docs/8.1/gist.html \tabularnewline
    https://www.postgresql.org/docs/16/brin-intro.html \tabularnewline
    https://www.postgresql.org/docs/current/sql-cluster.html \tabularnewline
    https://www.cybertec-postgresql.com/en/what-is-fillfactor-and-how-does-it-affect-postgresql-performance \tabularnewline
    \hline
  \end{tabular}
\end{table}

\end{document}
