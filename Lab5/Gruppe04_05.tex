\documentclass[11pt]{scrartcl}

\usepackage[top=1.5cm]{geometry}
\usepackage{float}

\setlength{\parindent}{0em}
\setlength{\parskip}{0.5em}

\newcommand{\youranswerhere}{[Your answer goes here \ldots]}
\renewcommand{\thesubsection}{\arabic{subsection}}

\title{
  \textbf{\large Assignment 5} \\
  Join Tuning \\
  {\large Database Tuning}}

\author{
  A4 \\
  \large Balint Peter, 12213073 \\
  \large Günter Lukas, 12125639 \\
  \large Ottino David, 51841010
}

\begin{document}

\maketitle

\subsection*{Experimental Setup}

Describe your experimental setup in a few lines.

We use a Python-script to run all of the tests in one go, measure the execution times and return the needed values. The queries are prepared once and the additional statements are added for each subsection of the tests to create indexes and force Postgres to use specific joins. We then drop all old indexes and run the subsection of queries prepared for each test.

\subsection*{Join Strategies Proposed by System}

\paragraph{Response times}\mbox{}

\begin{table}[H]
  \centering
  \begin{tabular}{l|l|l}
    Indexes & Join Strategy Q1 & Join Strategy Q2 \tabularnewline
    \hline
    no index & Hash Join (2.7250s) & Parallel Hash Join (0.3035s) \tabularnewline
    unique non-clustering on \texttt{Publ.pubID} & Hash Join (2.6926s) & Indexed Nested Loop Join (0.1715s)
      \tabularnewline
    clustering on \texttt{Publ.pubID} and \texttt{Auth.pubID} & Merge Join (2.4625s) & Indexed Nested Loop Join (0.2392s)
      \tabularnewline
  \end{tabular}
\end{table}

\paragraph{Discussion}

Discuss your observations. Is the choice of the strategy expected? How does the system come to this choice?

\textbf{Query 1}

With no indexes at all, PostgreSQL chooses to perform a hash join on
Publ.pubID (the side with the smaller number of tuples); this makes
sense, since hash joins are more performant than nested loop joins
without indexes when both sides have a large amount of tuples.

This does not change with a non-clustering index on Publ.  This is
probably because the number of hash lookups is significantly smaller in
this case than the number of index lookups that would be required when
using the non-clustering index.

Finally, when both sides have a clustering index each, the database
decides to run a merge join using the two indexes.  Evidently though,
this is not much more performant than the two alternatives; indeed, when
performing the experiment on an another system, PostgreSQL ended up
using a hash join instead.

\textbf{Query 2}

For the second query without an index, the optimizer starts a parallel
sequential scan on Auth, filling up the hash table with entries where
the name matches.  This is only possible here because the required hash
table is limited in size by the name filter; in a parallel hash, the
processes cooperate to create a huge hash table, which is not
realistically possible with millions of entries.

With a non-clustering index on Publ, we can also see a change: here,
the optimizer can finally use the index, since the tuples in Auth are
greatly filtered down according to the `name` column.

The strategy does not change with the two clustered indexes.  A merge
join would be possible here, but it is not used, presumably because it
would require a separate filter and sort step on the Auth index before
the scan on Publ could be started.

\subsection*{Indexed Nested Loop Join}

\paragraph{Response times}\mbox{}

\begin{table}[H]
  \centering
  \begin{tabular}{l|r|r}
    Indexes & Response time Q1 [s] & Response time Q2 [s] \tabularnewline
    \hline
    index on \texttt{Publ.pubID} & 13.6798 & 0.1692 \tabularnewline
    index on \texttt{Auth.pubID} & 5.7032 & 4.8619 \tabularnewline
    index on \texttt{Publ.pubID} and \texttt{Auth.pubID} & 5.3769 & 0.1867
        \tabularnewline
  \end{tabular}
\end{table}

\paragraph{Query plans}\mbox{}

Index on \texttt{Publ.pubID} (Q1/Q2):

{\small
\parskip0pt\begin{verbatim}
Creating non-clustered index on Publ.pubID
Gather  (cost=1000.43..987024.16 rows=3095200 width=82) (actual time=0.537..12599.057 rows=3095195 loops=1)
  Workers Planned: 2
  Workers Launched: 2
  ->  Nested Loop  (cost=0.43..676504.16 rows=1289667 width=82) (actual time=1.060..12261.941 rows=1031732 loops=3)
        ->  Parallel Seq Scan on auth  (cost=0.00..39705.67 rows=1289667 width=38) (actual time=0.435..175.958 rows=1031733 loops=3)
        ->  Index Scan using idx_publ_pubid_nonclustered on publ  (cost=0.43..0.48 rows=1 width=89) (actual time=0.011..0.011 rows=1 loops=3095200)  
              Index Cond: ((pubid)::text = (auth.pubid)::text)

Creating non-clustered index on Publ.pubID
Gather  (cost=1000.43..44016.78 rows=24 width=67) (actual time=36.762..173.169 rows=183 loops=1)
  Workers Planned: 2
  Workers Launched: 2
  ->  Nested Loop  (cost=0.43..43014.38 rows=10 width=67) (actual time=18.211..119.609 rows=61 loops=3)
        ->  Parallel Seq Scan on auth  (cost=0.00..42929.83 rows=10 width=23) (actual time=17.422..116.721 rows=61 loops=3)
              Filter: ((name)::text = 'Divesh Srivastava'::text)
              Rows Removed by Filter: 1031672
        ->  Index Scan using idx_publ_pubid_nonclustered on publ  (cost=0.43..8.45 rows=1 width=89) (actual time=0.045..0.046 rows=1 loops=183)      
              Index Cond: ((pubid)::text = (auth.pubid)::text)
\end{verbatim}}

Index on \texttt{Auth.pubID} (Q1/Q2):

{\small
\parskip0pt\begin{verbatim}
Creating non-clustered index on Auth.pubID
Gather  (cost=1000.43..662542.57 rows=3095200 width=82) (actual time=0.563..5169.208 rows=3095195 loops=1)
  Workers Planned: 2
  Workers Launched: 2
  ->  Nested Loop  (cost=0.43..352022.57 rows=1289667 width=82) (actual time=1.027..4927.963 rows=1031732 loops=3)
        ->  Parallel Seq Scan on publ  (cost=0.00..28618.39 rows=513839 width=89) (actual time=0.382..91.082 rows=411071 loops=3)
        ->  Index Scan using idx_auth_pubid_nonclustered on auth  (cost=0.43..0.60 rows=3 width=38) (actual time=0.010..0.011 rows=3 loops=1233213)  
              Index Cond: ((pubid)::text = (publ.pubid)::text)

Creating non-clustered index on Auth.pubID
Gather  (cost=1000.43..346601.98 rows=24 width=67) (actual time=399.018..4802.060 rows=183 loops=1)
  Workers Planned: 2
  Workers Launched: 2
  ->  Nested Loop  (cost=0.43..345599.58 rows=10 width=67) (actual time=359.002..4745.602 rows=61 loops=3)
        ->  Parallel Seq Scan on publ  (cost=0.00..28618.39 rows=513839 width=89) (actual time=0.425..101.265 rows=411071 loops=3)
        ->  Index Scan using idx_auth_pubid_nonclustered on auth  (cost=0.43..0.61 rows=1 width=23) (actual time=0.011..0.011 rows=0 loops=1233213)  
              Index Cond: ((pubid)::text = (publ.pubid)::text)
              Filter: ((name)::text = 'Divesh Srivastava'::text)
              Rows Removed by Filter: 3
\end{verbatim}}

Index on \texttt{Auth.pubID} and \texttt{Auth.pubID} (Q1/Q2):

{\small
\parskip0pt\begin{verbatim}
Creating non-clustered index on Publ.pubID
Creating non-clustered index on Auth.pubID
Gather  (cost=1000.43..662542.57 rows=3095200 width=82) (actual time=0.514..4953.691 rows=3095195 loops=1)
  Workers Planned: 2
  Workers Launched: 2
  ->  Nested Loop  (cost=0.43..352022.57 rows=1289667 width=82) (actual time=1.034..4714.075 rows=1031732 loops=3)
        ->  Parallel Seq Scan on publ  (cost=0.00..28618.39 rows=513839 width=89) (actual time=0.349..99.265 rows=411071 loops=3)
        ->  Index Scan using idx_auth_pubid_nonclustered on auth  (cost=0.43..0.60 rows=3 width=38) (actual time=0.010..0.011 rows=3 loops=1233213)  
              Index Cond: ((pubid)::text = (publ.pubid)::text)

Creating non-clustered index on Publ.pubID
Creating non-clustered index on Auth.pubID
Gather  (cost=1000.43..44016.78 rows=24 width=67) (actual time=45.347..188.893 rows=183 loops=1)
  Workers Planned: 2
  Workers Launched: 2
  ->  Nested Loop  (cost=0.43..43014.38 rows=10 width=67) (actual time=17.758..130.616 rows=61 loops=3)
        ->  Parallel Seq Scan on auth  (cost=0.00..42929.83 rows=10 width=23) (actual time=16.926..127.637 rows=61 loops=3)
              Filter: ((name)::text = 'Divesh Srivastava'::text)
              Rows Removed by Filter: 1031672
        ->  Index Scan using idx_publ_pubid_nonclustered on publ  (cost=0.43..8.45 rows=1 width=89) (actual time=0.047..0.047 rows=1 loops=183)      
              Index Cond: ((pubid)::text = (auth.pubid)::text)
\end{verbatim}}

\paragraph{Discussion}

Discuss your observations. Are the response times expected? Why (not)?

\textbf{Query 1}

When using a non-clustering index on Publ.pubID, the database performs a
sequential scan (which results in the longest runtime of all
experiments) and accesses the tuples from the Publ table based on the
index.  In contrast, with an index on Auth.pubID, Publ is read
sequentially, while the index is used for Auth.

As Auth with 3 million tuples is significantly larger than Publ with 1.3
million, it is generally more efficient to scan Publ completely.  This
assumption is confirmed by the shorter runtime in the corresponding
case.

If both tables are indexed, the system uses the index on Auth.pubID,
while the index on Publ is ignored.  The runtime is not noticeably
different from the variant in which only Auth.pubID is indexed - this
explains the optimizer's decision.

\textbf{Query 2}

If a non-clustering index is used on Publ.pubID, PostgreSQL runs through
the Auth table sequentially and filters for rows with \textit{name =
'Divesh Srivastava'}.  The corresponding tuple from Publ can then be
found very efficiently using the index via pubID - as this column is
unique.  This combination leads to a significantly better runtime
compared to query 1.

On the other hand, if there is only one index on Auth.pubID, PostgreSQL
decides to use the smaller Publ table as the outer table to perform an
indexed nested loop join, whereby the matching tuple from Auth is
searched for each row in Publ by index.  The name filter on Auth is
applied after the index access. Here too, the use of the index ensures a
better runtime, although not quite as efficiently as in the previous
case.

If both columns (Publ.pubID and Auth.pubID) are indexed, PostgreSQL only
uses the index on Publ.pubID in our case.  The additional index on Auth
remains unused.  The execution time is therefore not noticeably
different from the case in which only Publ.pubID is indexed.

\subsection*{Sort-Merge Join}

\paragraph{Response times}\mbox{}

\begin{table}[H]
  \centering
  \begin{tabular}{l|r|r}
    Indexes & Response time Q1 [s] & Response time Q2 [s] \tabularnewline
    \hline
    no index & 8.9741 & 1.4914 \tabularnewline
    two non-clustering indexes & 2.4118 & 0.7319 \tabularnewline
    two clustering indexes & 2.4007 & 0.7481 \tabularnewline
  \end{tabular}
\end{table}

\paragraph{Query plans}\mbox{}

No index (Q1/Q2):

{\small
\parskip0pt\begin{verbatim}
Gather  (cost=529161.49..867548.07 rows=3095200 width=82) (actual time=3735.644..6898.867 rows=3095195 loops=1)
  Workers Planned: 2
  Workers Launched: 2
  ->  Merge Join  (cost=528161.49..557028.07 rows=1289667 width=82) (actual time=3619.728..6181.447 rows=1031732 loops=3)
        Merge Cond: ((auth.pubid)::text = (publ.pubid)::text)
        ->  Sort  (cost=241129.63..244353.80 rows=1289667 width=38) (actual time=1292.754..1630.552 rows=1031733 loops=3)
              Sort Key: auth.pubid
              Sort Method: external merge  Disk: 54376kB
              Worker 0:  Sort Method: external merge  Disk: 47200kB
              Worker 1:  Sort Method: external merge  Disk: 46856kB
              ->  Parallel Seq Scan on auth  (cost=0.00..39705.67 rows=1289667 width=38) (actual time=0.498..107.316 rows=1031733 loops=3)
        ->  Materialize  (cost=287031.23..293197.30 rows=1233213 width=89) (actual time=2268.571..3327.095 rows=1860140 loops=3)
              ->  Sort  (cost=287031.23..290114.26 rows=1233213 width=89) (actual time=2268.552..3095.513 rows=1233127 loops=3)
                    Sort Key: publ.pubid
                    Sort Method: external merge  Disk: 121608kB
                    Worker 0:  Sort Method: external merge  Disk: 121608kB
                    Worker 1:  Sort Method: external merge  Disk: 121608kB
                    ->  Seq Scan on publ  (cost=0.00..35812.13 rows=1233213 width=89) (actual time=0.524..288.782 rows=1233213 loops=3)

Gather  (cost=170202.78..172771.80 rows=24 width=67) (actual time=1354.606..1675.787 rows=183 loops=1)
  Workers Planned: 2
  Workers Launched: 2
  ->  Merge Join  (cost=169202.78..171769.40 rows=10 width=67) (actual time=945.614..1362.222 rows=61 loops=3)
        Merge Cond: ((publ.pubid)::text = (auth.pubid)::text)
        ->  Sort  (cost=103702.98..104987.57 rows=513839 width=89) (actual time=509.776..796.875 rows=409986 loops=3)
              Sort Key: publ.pubid
              Sort Method: external merge  Disk: 47152kB
              Worker 0:  Sort Method: external merge  Disk: 38544kB
              Worker 1:  Sort Method: external merge  Disk: 35984kB
              ->  Parallel Seq Scan on publ  (cost=0.00..28618.39 rows=513839 width=89) (actual time=0.445..94.676 rows=411071 loops=3)
        ->  Sort  (cost=65499.55..65499.61 rows=24 width=23) (actual time=393.460..393.482 rows=183 loops=3)
              Sort Key: auth.pubid
              Sort Method: quicksort  Memory: 25kB
              Worker 0:  Sort Method: quicksort  Memory: 25kB
              Worker 1:  Sort Method: quicksort  Memory: 25kB
              ->  Seq Scan on auth  (cost=0.00..65499.00 rows=24 width=23) (actual time=31.243..393.273 rows=183 loops=3)
                    Filter: ((name)::text = 'Divesh Srivastava'::text)
                    Rows Removed by Filter: 3095017
\end{verbatim}}

Two non-clustering indexes (Q1/Q2):
{\small
\parskip0pt\begin{verbatim}
Creating non-clustered index on Publ.pubID
Creating non-clustered index on Auth.pubID
Merge Join  (cost=0.86..235103.52 rows=3095200 width=82) (actual time=0.018..2937.838 rows=3095195 loops=1)
  Merge Cond: ((publ.pubid)::text = (auth.pubid)::text)
  ->  Index Scan using idx_publ_pubid_nonclustered on publ  (cost=0.43..74420.80 rows=1233213 width=89) (actual time=0.005..439.313 rows=1233207 loops=1)
  ->  Index Scan using idx_auth_pubid_nonclustered on auth  (cost=0.43..118986.49 rows=3095200 width=38) (actual time=0.007..727.366 rows=3095200 loops=1)

Creating non-clustered index on Publ.pubID
Creating non-clustered index on Auth.pubID
Merge Join  (cost=43933.21..121360.18 rows=24 width=67) (actual time=258.439..1137.923 rows=183 loops=1)
  Merge Cond: ((publ.pubid)::text = (auth.pubid)::text)
  ->  Index Scan using idx_publ_pubid_nonclustered on publ  (cost=0.43..74420.80 rows=1233213 width=89) (actual time=0.008..416.211 rows=1229957 loops=1)
  ->  Sort  (cost=43932.78..43932.84 rows=24 width=23) (actual time=184.412..184.549 rows=183 loops=1)
        Sort Key: auth.pubid
        Sort Method: quicksort  Memory: 25kB
        ->  Gather  (cost=1000.00..43932.23 rows=24 width=23) (actual time=34.355..183.927 rows=183 loops=1)
              Workers Planned: 2
              Workers Launched: 2
              ->  Parallel Seq Scan on auth  (cost=0.00..42929.83 rows=10 width=23) (actual time=16.855..136.780 rows=61 loops=3)
                    Filter: ((name)::text = 'Divesh Srivastava'::text)
                    Rows Removed by Filter: 1031672
\end{verbatim}}

Two clustering indexes  (Q1/Q2):
{\small
\parskip0pt\begin{verbatim}
Creating clustered index on Publ.pubID
Creating clustered index on Auth.pubID
Merge Join  (cost=0.86..235103.52 rows=3095200 width=82) (actual time=0.033..3320.097 rows=3095195 loops=1)
  Merge Cond: ((publ.pubid)::text = (auth.pubid)::text)
  ->  Index Scan using idx_publ_pubid_clustered on publ  (cost=0.43..74420.80 rows=1233213 width=89) (actual time=0.010..473.425 rows=1233207 loops=1)
  ->  Index Scan using idx_auth_pubid_clustered on auth  (cost=0.43..118986.49 rows=3095200 width=38) (actual time=0.013..795.450 rows=3095200 loops=1)

Creating clustered index on Publ.pubID
Creating clustered index on Auth.pubID
Merge Join  (cost=43933.21..121360.18 rows=24 width=67) (actual time=320.572..1376.520 rows=183 loops=1)
  Merge Cond: ((publ.pubid)::text = (auth.pubid)::text)
  ->  Index Scan using idx_publ_pubid_clustered on publ  (cost=0.43..74420.80 rows=1233213 width=89) (actual time=0.010..493.428 rows=1229957 loops=1)
  ->  Sort  (cost=43932.78..43932.84 rows=24 width=23) (actual time=226.262..226.388 rows=183 loops=1)
        Sort Key: auth.pubid
        Sort Method: quicksort  Memory: 25kB
        ->  Gather  (cost=1000.00..43932.23 rows=24 width=23) (actual time=52.396..225.620 rows=183 loops=1)
              Workers Planned: 2
              Workers Launched: 2
              ->  Parallel Seq Scan on auth  (cost=0.00..42929.83 rows=10 width=23) (actual time=24.626..164.751 rows=61 loops=3)
                    Filter: ((name)::text = 'Divesh Srivastava'::text)
                    Rows Removed by Filter: 1031672
\end{verbatim}}

\paragraph{Discussion}

Discuss your observations. Are the response times expected? Why (not)?

\textbf{Query 1}

When using a merge join, the presence of non-clustered indexes on
Publ.pubID and Auth.pubID has no influence on the runtime.  In both
cases, the tables must first be explicitly sorted according to the
joined column before the merge join can be executed.  Non-clustered
indexes offer no advantage - neither during sorting nor during the
subsequent join.

The situation is different if there are clustered indexes on the joined
column in both tables.  In this case, the data is already sorted so that
the sorting effort is completely eliminated.  This results in a
significantly lower runtime, as the merge join can be executed directly
on the pre-sorted data.

\textbf{Query 2}

The second query exhibits similar behavior: if there is no clustering
index on the join columns Publ.pubID and \texttt{Auth.pubID}, both
tables must first be explicitly sorted before the merge join can be
performed.  The only difference to the first query is that Auth is
filtered to \textit{name = 'Divesh Srivastava'} before sorting.  This
reduces the amount of tuples joined, leading to a shorter runtime.

If both columns have a clustering index, Auth is also filtered by name
first.  As Publ is already sorted based on the clustering index, only
Auth needs to be sorted.  In this case, the runtime becomes
significantly shorter, as the join can be performed with barely any
sorting effort.

\subsection*{Hash Join}

\paragraph{Response times}\mbox{}

\begin{table}[H]
  \centering
  \begin{tabular}{l|r|r}
    Indexes & Response time Q1 [ms] & Response time [ms] Q2 \tabularnewline
    \hline
    no index & 2.6734 & 0.3059 \tabularnewline
  \end{tabular}
\end{table}

\paragraph{Query plans}\mbox{}

No Index (Q1/Q2):

{\small
\parskip0pt\begin{verbatim}
Hash Join  (cost=69292.29..236041.29 rows=3095200 width=82) (actual time=830.148..4453.022 rows=3095195 loops=1)
  Hash Cond: ((auth.pubid)::text = (publ.pubid)::text)
  ->  Seq Scan on auth  (cost=0.00..57761.00 rows=3095200 width=38) (actual time=0.068..454.953 rows=3095200 loops=1)
  ->  Hash  (cost=35812.13..35812.13 rows=1233213 width=89) (actual time=827.562..827.563 rows=1233213 loops=1)
        Buckets: 65536  Batches: 32  Memory Usage: 5111kB
        ->  Seq Scan on publ  (cost=0.00..35812.13 rows=1233213 width=89) (actual time=0.014..252.113 rows=1233213 loops=1)

Gather  (cost=43929.96..80258.37 rows=24 width=67) (actual time=226.941..344.091 rows=183 loops=1)
  Workers Planned: 2
  Workers Launched: 2
  ->  Parallel Hash Join  (cost=42929.96..79255.97 rows=10 width=67) (actual time=169.703..277.424 rows=61 loops=3)
        Hash Cond: ((publ.pubid)::text = (auth.pubid)::text)
        ->  Parallel Seq Scan on publ  (cost=0.00..28618.39 rows=513839 width=89) (actual time=0.210..58.086 rows=411071 loops=3)
        ->  Parallel Hash  (cost=42929.83..42929.83 rows=10 width=23) (actual time=156.991..156.992 rows=61 loops=3)
              Buckets: 1024  Batches: 1  Memory Usage: 104kB
              ->  Parallel Seq Scan on auth  (cost=0.00..42929.83 rows=10 width=23) (actual time=24.503..156.814 rows=61 loops=3)
                    Filter: ((name)::text = 'Divesh Srivastava'::text)
                    Rows Removed by Filter: 1031672
\end{verbatim}}

\paragraph{Discussion}

What do you think about the response time of the hash index vs.\ the response times of sort-merge and index nested loop join for each of the queries? Explain.

Overall, we can observe that without indexes, a hash join typically
achieves the best runtime.  With indexes, the optimal join strategy
strongly depends on whether the tables are clustered after the column to
be joined, and whether the number of tuples to be processed can be
restricted before the join.

\textbf{Query 1}

No selection is done in the first query - all tuples from Auth and Publ
must be joined.  In this case, the sort-merge join ends up being the
most efficient strategy, provided that both tables are clustered on the
join columns.  The existing sorting eliminates additional effort.

However, without clustering, PostgreSQL must sort both tables before the
merge, which leads to higher costs.  In this case, the hash join is
usually the better choice, as it does not require sorting and can handle
large amounts of data well.

Nested loop here is inefficient, mainly because Publ is being used as
the outer table.  Even with an index on \texttt{Auth}, it remains
significantly slower than merge and hash join.

Conclusion: a merge join with clustered indexes is the most efficient,
but hash join is the best option without indexes.

\textbf{Query 2}

The second query contains a selection condition on \texttt{Auth.name},
which filters down the amount of tuples by several orders of magnitude.
An indexed nested loop join in particular can benefit from this.  This
is recognized by the PostgreSQL planner as well.

If no suitable index is available, the hash join is once again a
reasonably efficient option.  The merge join with clustered indexes on
the other hand is still clearly slower than the aforementioned two
arrangements.

Conclusion: for selective conditions on a table, the nested loop with
index on the counter-table (here Publ) is clearly superior.

\subsection*{Time Spent on this Assignment}

Time in hours per person: \textbf{4}

\subsection*{References}

\begin{table}[H]
  \centering
  \begin{tabular}{c}
    \hline
    \tabularnewline
    https://www.enterprisedb.com/postgres-tutorials/parallel-hash-joins-postgresql-explained?lang=en
  \end{tabular}
\end{table}

\end{document}
