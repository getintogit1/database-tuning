\documentclass[11pt]{scrartcl}

\usepackage[top=1.5cm]{geometry}
\usepackage{float}

\setlength{\parindent}{0em}
\setlength{\parskip}{0.5em}

\newcommand{\youranswerhere}{[Your answer goes here \ldots]}
\renewcommand{\thesubsection}{\arabic{subsection}}

\title{
  \textbf{\large Assignment 5} \\
  Join Tuning \\
  {\large Database Tuning}}

\author{
  Group Name (e.g. A1, B5, B3) \\
  \large Lastname1 Firstname1, StudentID1 \\
  \large Lastname2 Firstname2, StudentID2 \\
  \large Lastname3 Firstname3, StudentID3
}

\begin{document}

\maketitle

\subsection*{Experimental Setup}

Describe your experimental setup in a few lines.

\youranswerhere{}

\subsection*{Join Strategies Proposed by System}

\paragraph{Response times}\mbox{}

\begin{table}[H]
  \centering
  \begin{tabular}{l|l|l}
    Indexes & Join Strategy Q1 & Join Strategy Q2 \tabularnewline
    \hline
    no index & \ldots & \ldots \tabularnewline
    unique non-clustering on \texttt{Publ.pubID} & \ldots  & \ldots
      \tabularnewline
    clustering on \texttt{Publ.pubID} and \texttt{Auth.pubID} & \ldots & \ldots
      \tabularnewline
  \end{tabular}
\end{table}

\paragraph{Discussion}

Discuss your observations. Is the choice of the strategy expected? How does the system come to this choice?

\youranswerhere{}

\subsection*{Indexed Nested Loop Join}

\paragraph{Response times}\mbox{}

\begin{table}[H]
  \centering
  \begin{tabular}{l|r|r}
    Indexes & Response time Q1 [ms] & Response time Q2 [ms] \tabularnewline
    \hline
    index on \texttt{Publ.pubID} & \ldots & \ldots \tabularnewline
    index on \texttt{Auth.pubID} & \ldots & \ldots \tabularnewline
    index on \texttt{Publ.pubID} and \texttt{Auth.pubID} & \ldots & \ldots
        \tabularnewline
  \end{tabular}
\end{table}

\paragraph{Query plans}\mbox{}

Index on \texttt{Publ.pubID} (Q1/Q2):

{\small
\parskip0pt\begin{verbatim}
[Your query plans (index on Publ.pubID) go here ...]
\end{verbatim}}

Index on \texttt{Auth.pubID} (Q1/Q2):

{\small
\parskip0pt\begin{verbatim}
[Your query plans (index on Auth.pubID) go here ...]
\end{verbatim}}

Index on \texttt{Auth.pubID} and \texttt{Auth.pubID} (Q1/Q2):

{\small
\parskip0pt\begin{verbatim}
[Your query plans (index on Publ.pubID and Auth.pubID) go here ...]
\end{verbatim}}

\paragraph{Discussion}

Discuss your observations. Are the response times expected? Why (not)?

\youranswerhere{}

\subsection*{Sort-Merge Join}

\paragraph{Response times}\mbox{}

\begin{table}[H]
  \centering
  \begin{tabular}{l|r|r}
    Indexes & Response time Q1 [ms] & Response time Q2 [ms] \tabularnewline
    \hline
    no index & \ldots & \ldots \tabularnewline
    two non-clustering indexes & \ldots & \ldots \tabularnewline
    two clustering indexes & \ldots & \ldots \tabularnewline
  \end{tabular}
\end{table}

\paragraph{Query plans}\mbox{}

No index (Q1/Q2):

{\small
\parskip0pt\begin{verbatim}
[Your query plans (no index) go here ...]
\end{verbatim}}

Two non-clustering indexes (Q1/Q2):
{\small
\parskip0pt\begin{verbatim}
[Your query plans (two non-clustering indexes) go here ...]
\end{verbatim}}

Two clustering indexes  (Q1/Q2):
{\small
\parskip0pt\begin{verbatim}
[Your query plans (two clustering indexes) go here ...]
\end{verbatim}}

\paragraph{Discussion}

Discuss your observations. Are the response times expected? Why (not)?

\youranswerhere{}

\subsection*{Hash Join}

\paragraph{Response times}\mbox{}

\begin{table}[H]
  \centering
  \begin{tabular}{l|r|r}
    Indexes & Response time Q1 [ms] & Response time [ms] Q2 \tabularnewline
    \hline
    no index & \ldots & \ldots \tabularnewline
  \end{tabular}
\end{table}

\paragraph{Query plans}\mbox{}

No Index (Q1/Q2):

{\small
\parskip0pt\begin{verbatim}
[Your query plans (no index) go here ...]
\end{verbatim}}

\paragraph{Discussion}

What do you think about the response time of the hash index vs.\ the response times of sort-merge and index nested loop join for each of the queries? Explain.

\youranswerhere{}

\subsection*{Time Spent on this Assignment}

Time in hours per person: \textbf{XXX}

\subsection*{References}

\begin{table}[H]
  \centering
  \begin{tabular}{c}
    \hline
    \textbf{Important:} Reference your information sources! \tabularnewline
    Remove this section if you use footnotes to reference your information
    sources. \tabularnewline
    \hline
  \end{tabular}
\end{table}

\end{document}
